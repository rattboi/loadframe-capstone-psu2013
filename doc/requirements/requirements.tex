%%MaD.tex - Notes taken for Materials and Devices Lecture
%%Author: Andy Goetz
%%Date Modified: 10-7-09
%%License: Ask me before reproducing/modifying, etc.


\documentclass{article}

%Make sure you have the file ShumanNote.scy in the same directory as
%this one. It has contains the style sheet for ECE111, and is needed
%to standardize the layout of LateX documents created for the class.
\usepackage{ShumanNotes} 
\usepackage{tikz}
\usepackage{program}
\usepackage{listings}
\lstset{
  showstringspaces=false
}
\pdfpagewidth 8.5in 
\pdfpageheight 11in

%This package is used to line up pictures 
\usepackage{graphicx}
\usepackage{fancyvrb}
\usepackage{listings}
%allows cursive font
%\usepackage{amsmath}

%allows hyperlinks 
\usepackage{hyperref}

\newcommand{\HRule}{\rule{\linewidth}{0.5mm}} 

\lhead{\leftmark}
\lfoot{Requirements}
\begin{document}



%% These commands allow me to use cursive letter for things such as
%% length.  Note that on ubuntu linux, this required installation of
%% the package 'texlive-fonts-extra'. 
%% Taken from
%% http://www.latex-community.org/forum/viewtopic.php?f=5&t=1404&start=0
\newenvironment{frcseries}{\fontfamily{frc}\selectfont}{}
\newcommand{\textfrc}[1]{{\frcseries#1}}
\newcommand{\mathfrc}[1]{\text{\textfrc{#1}}}

\setcounter{tocdepth}{2}

\begin{titlepage}
 
\begin{center}
 
 
\textsc{\LARGE MTS Load Frame Capstone}\\[1.5cm]
 
\textsc{\Large Portland State University}\\[0.5cm]
 
 
% Title
\HRule \\[0.4cm]
{ \huge \bfseries Requirements}\\[0.4cm]
 
\HRule \\[1.5cm]
 
% Author and supervisor
\begin{minipage}{0.4\textwidth}
\begin{center} \large
Andy \textsc{Goetz}, Bradon \textsc{Kanyid}, Mikhail \textsc{Kulakevich}\\
\hspace{3\textheight}
\emph{Advisor:} Dr. Richard \textsc{Tymerski}
\end{center}
\end{minipage}

 

 
 
\end{center}
\vfill
{ \textit{} }\\[4.0cm]
\begin{center}
% Bottom of the page
{\large \today}

\end{center} 
\end{titlepage}

\newpage

\textbf{\large Version}
\vspace{.1in}

\begin{tabular}{|p{1in}|p{4in}|p{1in}|}
\hline
\multicolumn{3}{|l|}{\textbf{Document Review}} \\ 
\hline
\textbf{Version}& \textbf{Comments} & \textbf{Date} \\ 
\hline
\textbf{1.0} & Initial release. & 1/25/2013 \\
\hline
\end{tabular}

\newpage


\section{Overview}

The purpose of this document is to lay out the requirements for the
MTS Load Frame Control and Logging Capstone project. In general terms,
the scope of this project involves integrating the 22 kip Load Frame
in the EB370 Lab with an existing hydraulic pump and
servocontroller. Additionally, the project will create a user-friendly
interface, complete with data logging and safety interlocks. 

\newpage

\tableofcontents

\newpage

\section{Requirements}

This document is organized in to a collection of individual
requirements, in no specific order. Detailed specifications of these
requirements will be part of the next version of this document. 

\subsection{Safety}
Because of the large forces inherent in this project, there are
several safety requirements that are perhaps unfamiliar to ECE
students. 

\subsubsection{Emergency Stop} 
The system must contain an emergency stop button that immediately
prevents the movement of the load frame ram. 

\subsubsection{Limit detection}
The system must have some way of limiting the movement of the load
frame ram from extending past its minimum and maximum position. 

\subsection{User Interface}

The user interface section of this document describes both the
requirements for the user interface, as well as the requirements for
the mechanical of the load frame in general.

\subsubsection{Status Display} 

The user interface must display the current status of the load frame,
as well as the position and load on the load cell at all times.

\subsubsection{Frame Controls} 

The user interface must allow the load frame to be manually moved up
and down, as well as being automatically moved.

\subsection{Data Logging}

In order to allow the load frame to be used for educational purposes,
the load frame must have detailed data logging capabilities. 

\subsubsection{Labview Compatibility}

In order to take advantage of existing teaching methodologies, the
load frame data logging features must be useable from the National
Instruments Labview Software. 

\subsubsection{Load and Position Logging}

The position and load on the load cell must be capable of being logged. 

\subsubsection{Clip-on and Extensiometer Logging}

The load frame must be capable of logging the output of up to 4
clip-on sensors simultaneously, as well as a laser extensiometer.

\subsection{Budget}
This project has a maximum budget of \$1000 dollars, not including
parts already purchased, including the MTS Load Frame, the servo
controller, and hydraulic pump. 

\end{document}

